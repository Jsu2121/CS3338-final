\documentclass[12pt]{article}
%---- packages -----
\usepackage{geometry}
\usepackage{graphicx}
\usepackage{fancyhdr}
\usepackage{hyperref}
% \usepackage{array}
% \usepackage{longtable}
% \usepackage{titling}

%---- page setup -----
\geometry{a4paper, margin=1in}
\pagestyle{fancy}
\fancyhead[L]{AI-Powered Triage System - System Design Document (SDD)}
\fancyhead[R]{Page \thepage}
\fancyfoot[C]{}

%---- title -----
\title{AI-Powered Triage System \\
System Design Document (SDD)}
\author{
Group 17\\
Esmeralda Amado\\
Jessica Parker\\
Jonathan Su\\
Nneoma Ajawara\\
Yvan Michel Kemsseu Yobeu
}

\date{December 2025}

\begin{document}

\begin{titlepage}
    \maketitle
    \thispagestyle{empty}
\end{titlepage}

\thispagestyle{empty}
\tableofcontents
\newpage

%---------------------------
\section*{Version Description}
\addcontentsline{toc}{section}{Version Description}

\begin{table}
    \begin{tabular}{|c|p{8cm}|c|}
    \hline
    \textbf{Version} & \textbf{Description} & \textbf{Date Added} \\ \hline
    0.1 & Initial SDD created for Start Snapshot and planned system architecture. & 11/30/2025 \\ \hline
    0.2 & Added UI design and database overview. & 12/01/2025 \\ \hline
    1.0 & Final version for Start Snapshot submission. & 12/05/2025 \\ \hline
    \end{tabular}
\end{table}

\newpage

%--------------------------
\section{Introduction}

\subsection{Purpose}
The purpose of this Software Design Document (SDD) is to outline the technical design and system structure for the AI-Powered Triage System. This version covers the Start Snapshot, representing the foundation of the system.

\subsection{Intended Audience}
This document is intended for:
\begin{itemize}
    \item The CS 3338 course instructor and graders.
    \item Developers responsible for implementing system components.
    \item Testers working with TestRail and Jira.
    \item Future maintainers/contributors of the system, in later snapshots.
\end{itemize}

\subsection{System Overview}
The AI-Powered Triage System is a web-based healthcare tool that allows  authorized medical staff to:
\begin{itemize}
    \item Log in securely.
    \item Access patient demographic and historical visit information.
    \item Receive AI-generated summaries and triage recommendations.
    \item Complete medical visit documentation assisted by AI.
\end{itemize}
This SDD describes the structure and behavior of the sytem and all major components.

\newpage

%--------------------------
\section{System Architecture}

\subsection{Workflow}
The workflow of the AI Triage System is as follows:
\begin{enumerate}
    \item The user logs in through the secure authentication portal.
    \item The front-end client requests patient data from the back-end API.
    \item Data is retrieved from the database and displayed to the user.
    \item When the user interacts with the AI, the back-end sends relevant patient information to the AI model.
    \item The AI returns a summary or triage recommendation.
    \item The user updates the patient's visit history or completes forms.
    \item The system stores updated visit documentation in the database.
\end{enumerate}

\subsection{Component Breakdown}
The AI Triage System will consist of the following components:
\begin{itemize}
    \item \textbf{Client-Side (Front-End)}
    A web interface built using HTML/CSS/JavaScript or a front-end framework like React.
    
    \item \textbf{Server-Side (Back-End)}
    A REST API built using Python Flask, FastAPI, or Node.js for:
    \begin{itemize}
        \item User authentication
        \item Database interactions
        \item Passing data to/from the AI model
    \end{itemize}
    
    \item \textbf{Database Layer}
    Will store patient demographic information and historic visit notes using a relational database (MySQL).
    
    \item \textbf{AI Service Layer}
    Connects to an AI model such as OpenAI GPT or a local machine learning model trained on healthcare text.
\end{itemize}

\new page

%--------------------------
\section{User Interface}

\subsection{How to Use}
Users will be able to interact with the system, using the following components:
\begin{itemize}
    \item \textbf{Main UI Pages}: Verified healthcare staff can authenticate themselves through the Login Page. They can also view patients' demographics and visit history through the Patient Profile Page.
    \item \textbf{Dashboard}: To display list of patients, recent activity, and quick links.
    \item \textbf{AI Chat Interface}: Allows staff to ask triage-related questions.
    \item \textbf{Visit Documentation Form}: A structured form for healthcare staff to complete patient encouter records. 
\end{itemize}

\subsection{Database Design Overview}
Database will contain the following tables:
\begin{itemize}
    \item \textbf{Staff}
    Columns: staff\_id, name, email, role, password\_hash
    \item \textbf{Patients}
    Columns: patient\_id, name, age, gender, date\_of\_birth
    \item \textbf{Visits}
    Columns: visit\_id, patient\_id, visit\_date, condition, treatment, outcome
\end{itemize}

\subsection{UI Screenshots}
Screenshots of tentative UI design may be added in later versions.

\newpage

%--------------------------
\section{Glossary}
% \addcontentsline{toc}{section}{Glossary}
\begin{description}
    \item[\textbf{AI}] - Artificial Intelligence
    \item[\textbf{API}] - Application Programming Interface
    \item[\textbf{DB}] - Database
    \item[\textbf{SDD}] - System Design Document
    \item[\textbf{UI}]- User Interface
\end{description}

\newpage

%--------------------------
\section{References}
Current references:
\begin{itemize}
    \item CS 3338 Lab 14 Instructions (Canvas).
    \item CS 3338 Final Project Instructions (Canvas).
    \item Example Final Projects From Past Group (Github).
    \item Overleaf Documentation: \url{https://www.overleaf.com/learn}
    \item HIPAA Overview: \url{https://www.hhs.gov/hipaa/}
\end{itemize}

\end{document}
