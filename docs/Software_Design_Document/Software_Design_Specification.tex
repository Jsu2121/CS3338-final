% group17_SDD.tex
\documentclass[12pt]{article}

\usepackage[utf8]{inputenc}
\usepackage[letterpaper,margin=1in]{geometry}
\usepackage{setspace}
\usepackage{hyperref}
\usepackage{array}

\title{Software Design Document (SDD)\\
\large AI-Powered Triage System}

\author{
CS 3338 -- Group 17\\[8pt]
Esmeralda Amado\\
Jessica Parker\\
Jonathan Su\\
Nneoma Ajawara\\
Yvan Michel Kemsseu Yobeu
}

\date{December 2025}

\begin{document}
\maketitle
\onehalfspacing

\tableofcontents
\newpage

% ============================================================
\section{Version Description}

\begin{table}[h]
  \centering
  \begin{tabular}{|c|p{8cm}|c|}
    \hline
    \textbf{Version} & \textbf{Description} & \textbf{Date} \\ \hline
    0.1 & Initial creation of SDD layout and core system architecture. & Nov 30, 2025 \\ \hline
    0.2 & Added UI design, database overview, and workflow explanation. & Dec 1, 2025 \\ \hline
    1.0 & Final version for Snapshot 1 submission. & Dec 5, 2025 \\ \hline
  \end{tabular}
  \caption{SDD Version Description}
\end{table}

\newpage

% ============================================================
\section{Introduction}

\subsection{Purpose}
The purpose of this Software Design Document (SDD) is to provide the technical
design and system architecture for the AI-Powered Triage System. This document
serves as a guide for developers, testers, and future team members.

\subsection{Intended Audience}
This document is intended for:
\begin{itemize}
    \item The CS 3338 course instructor and graders.
    \item Developers responsible for implementing system components.
    \item Testers working with TestRail and Jira.
    \item Future maintainers of the AI triage platform.
\end{itemize}

\subsection{System Overview}
The AI-Powered Triage System is a web-based healthcare tool that allows
authorized medical staff to:
\begin{itemize}
    \item Log in securely.
    \item Access patient demographic and historical visit information.
    \item Receive AI-generated summaries and triage recommendations.
    \item Complete medical visit documentation assisted by AI.
\end{itemize}

This SDD describes the structure and behavior of the system and all major components.

% ============================================================
\section{System Architecture}

\subsection{System Workflow}
The high-level workflow of the system is as follows:

\begin{enumerate}
    \item The user logs in through the secure authentication portal.
    \item The front-end client requests patient data from the back-end API.
    \item Data is retrieved from the database and displayed to the user.
    \item When the user interacts with the AI, the back-end sends relevant
    patient information to the AI model.
    \item The AI returns a summary or triage recommendation.
    \item The user updates the patient's visit history or completes forms.
    \item The system stores updated visit documentation in the database.
\end{enumerate}

\subsection{Architecture Diagram Description}
The system contains the following functional layers:

\begin{itemize}
    \item \textbf{Client-Side (Front-End)}  
    A web interface built using HTML/CSS/JavaScript or a front-end framework
    such as React or Vue.

    \item \textbf{Server-Side (Back-End)}  
    A REST API built using Python Flask, FastAPI, or Node.js. Handles:
    \begin{itemize}
        \item User authentication
        \item Database interactions
        \item Passing data to/from the AI model
    \end{itemize}

    \item \textbf{Database Layer}  
    Stores patient demographic information and historic visit notes using a
    relational database (PostgreSQL or MySQL).

    \item \textbf{AI Service Layer}  
    Connects to an AI model such as OpenAI GPT or a local machine learning
    model trained on healthcare text.
\end{itemize}

% ============================================================
\section{User Interface Design}

\subsection{Overview of User Interface}
The user interface will consist of the following pages:

\begin{itemize}
    \item \textbf{Login Page}: Allows verified healthcare staff to authenticate.
    \item \textbf{Dashboard}: Displays a list of patients, recent activity,
    and quick links.
    \item \textbf{Patient Profile Page}: Shows demographics and visit history.
    \item \textbf{AI Chat Interface}: Allows staff to ask triage-related questions.
    \item \textbf{Visit Documentation Form}: A structured form for completing
    patient encounter records.
\end{itemize}

\subsection{Database Design Overview}
The database will contain the following example tables:

\begin{itemize}
    \item \textbf{Users}  
    Columns: user\_id, name, email, role, password\_hash

    \item \textbf{Patients}  
    Columns: patient\_id, name, age, gender, date\_of\_birth

    \item \textbf{Visits}  
    Columns: visit\_id, patient\_id, visit\_date, condition, treatment, outcome

    \item \textbf{ChatLogs} (optional)  
    Columns: chat\_id, user\_id, patient\_id, timestamp, message
\end{itemize}

\subsection{Using the System}
Users interact with the system by:
\begin{itemize}
    \item Logging in with secure credentials.
    \item Selecting a patient record from the dashboard.
    \item Opening the AI assistant to obtain summary information.
    \item Completing triage documentation with AI-assisted text suggestions.
\end{itemize}

\subsection{Optional UI Screenshots}
Because the system is still in Snapshot 1, UI screenshots may not be final.  
Screenshots will be added as development progresses.

% ============================================================
\section{Glossary}

\begin{table}[h]
  \centering
  \begin{tabular}{|l|p{9cm}|}
    \hline
    \textbf{Acronym} & \textbf{Definition} \\ \hline
    AI & Artificial Intelligence. \\ \hline
    UI & User Interface. \\ \hline
    API & Application Programming Interface. \\ \hline
    DB & Database storing patient and visit data. \\ \hline
    SRS & Software Requirements Specification. \\ \hline
    SDD & Software Design Document. \\ \hline
  \end{tabular}
  \caption{Glossary of Acronyms}
\end{table}

% ============================================================
\section{References}

\begin{itemize}
    \item CS 3338 Final Project Instructions (Canvas).
    \item Overleaf Documentation: \url{https://www.overleaf.com/learn}
    \item PostgreSQL Documentation: \url{https://www.postgresql.org/docs/}
    \item HIPAA Overview: \url{https://www.hhs.gov/hipaa/}
\end{itemize}

% ============================================================
\section{Appendix}

This document represents Version 1.0 for Snapshot 1 and will be updated as the
AI-Powered Triage System evolves through additional snapshots.

\end{document}