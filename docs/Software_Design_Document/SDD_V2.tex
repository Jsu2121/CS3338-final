\documentclass[12pt]{article}
%---- packages -----
\usepackage{geometry}
\usepackage{graphicx}
\usepackage{fancyhdr}
\usepackage{hyperref}
% \usepackage{array}
% \usepackage{longtable}
% \usepackage{titling}

%---- page setup -----
\geometry{a4paper, margin=1in}
\pagestyle{fancy}
\fancyhead[L]{AI-Powered Triage System - System Design Document (SDD)
 V.2}
\fancyhead[R]{Page \thepage}
\fancyfoot[C]{}

%---- title -----
\title{AI-Powered Triage System \\
System Design Document (SDD) \\
Version 2}
\author{
Group 17\\
Esmeralda Amado\\
Jessica Parker\\
Jonathan Su\\
Nneoma Ajawara\\
Yvan Michel Kemsseu Yobeu
}

\date{December 2025}

\begin{document}

\begin{titlepage}
    \maketitle
    \thispagestyle{empty}
\end{titlepage}

\thispagestyle{empty}
\tableofcontents
\newpage

%---------------------------
\section*{Version Description}
\addcontentsline{toc}{section}{Version Description}

\begin{table}
    \begin{tabular}{|c|p{8cm}|c|}
    \hline
    \textbf{Version} & \textbf{Description} & \textbf{Date Added} \\ \hline
    1.0 & Final version for Start Snapshot submission. & 12/05/2025 \\ \hline
    2.0 & Added full database design, updated workflow, added CRUD operations and implemented core triage features. & 12/08/2025 \\ \hline
    \end{tabular}
\end{table}

\newpage

%--------------------------
\section{Introduction}

\subsection{Purpose}
The purpose of Version 2 of this Software Design Document (SDD) is to describe
the newly implemented database design and core triage workflow for the AI-Powered
Triage System. This version expands on the initial framework introduced in Version 1.

\subsection{Intended Audience}
This document is intended for:
\begin{itemize}
    \item The instructor and the graders of the CS 3338 course.
    \item Developers responsible for implementing system components.
    \item Testers working with TestRail and Jira.
    \item Future maintainers/contributors of the system, in later snapshots.
\end{itemize}

\subsection{System Overview}
The AI-Powered Triage System now includes a defined database structure and a functional workflow for:
\begin{itemize}
    \item Storing patient information.
    \item Retrieving patient information.
\end{itemize}
The AI system is still not implemented, but the system architecture has been expanded to support patient data entry and basic triage record management.

\newpage

%--------------------------
\section{System Architecture}

\subsection{Workflow}
The workflow of the AI Triage System is as follows:
\begin{enumerate}
    \item The user logs in through the secure authentication portal.
    \item The front-end client requests patient data from the back-end API.
    \item Data is retrieved from the database and displayed to the user.
    \item The user selects a patient or creates a new patient record.
    \item The user updates the patient's visit history or completes forms.
    \item The user can later retrieve and review stored visit information (notes).
\end{enumerate}

\subsection{Component Breakdown}
The AI Triage System will consist of the following components:
\begin{itemize}
    \item \textbf{Client-Side (Front-End)}
    A web interface built using HTML/CSS/JavaScript or a front-end framework like React.
    
    \item \textbf{Server-Side (Back-End)}
    A REST API built using Python Flask, FastAPI, or Node.js for:
    \begin{itemize}
        \item User authentication
        \item Database interactions
        \item Passing data to/from the AI model
    \end{itemize}
    
    \item \textbf{Database Layer}
    A relational database (MySQL) has been defined for storing patient demographics and visit records. The database doesn't interact with AI yet, but supports basic database organization and retrieval.
    
    \item \textbf{AI Service Layer}
    This is not yet implemented and will be introduced in future versions.
    % Connects to an AI model such as OpenAI GPT or a local machine learning model trained on healthcare text.
\end{itemize}

\newpage

%--------------------------
\section{User Interface}

\subsection{How to Use}
Users will be able to interact with the system, using the following components:
\begin{itemize}
    \item \textbf{Patient Profile Page}: Verified healthcare personnel can also enter visit notes, and view their visit history.
    \item \textbf{Dashboard}: To display a list of patients, recent activity, and quick links.
    \item \textbf{Visit Documentation Form}: A structured form for healthcare personnel to complete patient encounter records. 
\end{itemize}

\subsection{Database Design Overview}
The database will contain the following tables:
\begin{itemize}
    \item \textbf{Staff}
    Columns: staff\_id, name, email, role, password\_hash
    \item \textbf{Patients}
    Columns: patient\_id, name, age, sex, date\_of\_birth
    \item \textbf{Visits}
    Columns: visit\_id, patient\_id, visit\_date, condition, treatment, outcome
\end{itemize}

\subsection{UI Screenshots}
Screenshots of the tentative UI design may be added in later versions.

\newpage

%--------------------------
\section{Glossary}
% \addcontentsline{toc}{section}{Glossary}
\begin{description}
    \item[\textbf{AI}] - Artificial Intelligence
    \item[\textbf{API}] - Application Programming Interface
    \item[\textbf{DB}] - Database
    \item[\textbf{SDD}] - System Design Document
    \item[\textbf{UI}]- User Interface
\end{description}

\newpage

%--------------------------
\section{References}
Current references:
\begin{itemize}
    \item CS 3338 Lab 14 Instructions (Canvas).
    \item CS 3338 Final Project Instructions (Canvas).
    \item Example Final Projects From Past Group (Github).
    \item Overleaf Documentation: \url{https://www.overleaf.com/learn}
    \item HIPAA Overview: \url{https://www.hhs.gov/hipaa/}
\end{itemize}

\end{document}
