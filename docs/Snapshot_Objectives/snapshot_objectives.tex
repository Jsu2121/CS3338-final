\documentclass{article}

\usepackage{geometry}
\usepackage{fancyhdr}

\geometry{a4paper, margin=1in}

\title{Snapshot Objectives}
\author{
    Esmeralda Amado\\
    \and
    Jessica Parker\\
    \and
    Jonathan Su\\
    \and
    Nneoma Ajawara\\
    \and
    Yvan Michel Kemsseu Yobeu
}
\date{\today}

\pagestyle{fancy}
\fancyhead[L]{Snapshot Objectives}
\fancyhead[C]{}
\fancyhead[R]{\thepage}

\begin{document}

\begin{titlepage}
\maketitle
\end{titlepage}

\section{Start}
The starting objective of our AI triage system project is to develop the necessary frontend and backend framework to support our future development.
The initial focus will be on creating a user-friendly interface for users to navigate and interact with the system. This includes designing intuitive layouts, implementing responsive design principles, and ensuring accessibility for all users. Our frontend development will allow for easy modernization and scalability as we add more features in the future.
On the backend, we will establish a robust infrastructure to handle data storage, retrieval, and processing. This will involve setting up databases, APIs, and server-side logic to support the application's functionality.
Given the importance of these frameworks, we will prioritize their development to ensure a solid foundation for the AI triage system. The majority of the team will focus on these tasks, while a smaller group will begin preliminary work on future features.

\section{Checkpoint 1}
With a functioning frontend and backend framework in place, our next objective is to implement the core features of the AI triage system. This includes setting up the database collections to store user data, medical records, and triage information. It is important the database is well-structured and optimized for efficient data retrieval and storage as well as securied to protect sensitive information.
This task will be primarily handled by a large subgroup of the team, allowing the rest of the members to focus on specialized individual testing for system performance, bug fixes, security, etc.

\section{Checkpoint 2}
With a solid database structure established, our next objective is to integrate AI functionalities into the triage system. This involves developing and implementing machine learning algorithms that can analyze patient symptoms and medical history to provide accurate triage recommendations. In addition, we will work on enhancing the user interface to ensure seamless interaction with the AI features. This means refining the design, improving usability, and ensuring that users can easily access and understand the AI-generated recommendations.
This phase will require prolong discussion with the entire team to ensure that the AI integration aligns with our overall project goals and user needs. The AI should be thoroughly tested to ensure its accuracy and reliability in real-world scenarios, securied to protect patient data, and optimized for performance to handle large volumes of data efficiently.

\section{Due Date Checkpoint}
By the end of development, we plan to have a fully functional AI triage system that meets all project objectives. This includes a user-friendly interface, robust backend infrastructure, secure database management, and accurate AI-driven triage recommendations.
We aim to create a system that not only meets the technical requirements but also allow for easy adoption of emerging technologies in the future. This means designing the system with scalability and flexibility in mind, allowing for the integration of new features and improvements as technology evolves.
Some plans for future features, beyond the scope of this development cycle, would include a camara that allows the AI to recognize patient, their physical symptoms, and diagnose patients.

\end{document}