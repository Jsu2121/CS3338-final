\documentclass{article}

\usepackage{geometry}
\usepackage{fancyhdr}
\usepackage{hyperref}
\usepackage{graphicx}

\geometry{a4paper, margin=1in}

\title{User Manual}
\author{
    CS3338 Final Project Team\\
    \and
    Esmeralda Amado\\
    \and
    Jessica Parker\\
    \and
    Jonathan Su\\
    \and
    Nneoma Ajawara\\
    \and
    Yvan Michel Kemsseu Yobeu
}
\date{\today}

\begin{document}

\begin{titlepage}
\maketitle
\end{titlepage}

\section{Jira Project Link}
\href{https://calstatela-cs3338-group17final.atlassian.net/jira/software/projects/SCRUM/summary}


\section{Object Breakdown}
The objective of this project is to create an application that allows healthcare professionals to efficiently manage patient history, records, and documentation.
This application aims to streamline the process of accessing and updating patient information, ensuring that healthcare providers have accurate and up-to-date data at their fingertips. By implementing this system, we hope to enhance the quality of care provided to patients while reducing administrative burdens on medical staff.

\section{Goals}
The primary goals of this project are as follows:
\begin{itemize}
    \item Develop a user-friendly interface for healthcare professionals to access and manage patient records.
    \item Implement secure data storage and retrieval mechanisms to protect patient information.
    \item Provide search and filter functionalities to quickly locate specific patient records.
    \item Integrate an AI system that can assist in summarizing patient history and suggesting potential diagnoses based on recorded symptoms and treatments.
\end{itemize}

\section{How to Access the Application}
\subsection{Prerequisites}
Before accessing the application, ensure that you have the following prerequisites:
\begin{itemize}
    \item Healthcare professional credentials (e.g., doctor, nurse, medical assistant).
    \item Permission from the healthcare institution to access AI-assisted patient management system.
\end{itemize}
\subsection{Accessing the Application}
To access the application, follow these steps:
\begin{itemize}
    \item Open your web browser and navigate to the application URL: \url{http://healthcare-app.example.com}.
    \item Download and install the application from the provided link if it is a desktop or mobile application.
    \item Launch the application and log in using your healthcare professional credentials.
    \item Once logged in, you will be directed to the main dashboard where you can start managing patient records.
\end{itemize}