% group17_SRS.tex  
\documentclass[12pt]{article}

\usepackage[utf8]{inputenc}
\usepackage[letterpaper,margin=1in]{geometry}
\usepackage{setspace}
\usepackage{hyperref}
\usepackage{array}

\title{Software Requirements Specification (SRS)\\
\large AI-Powered Triage System}

\author{
CS 3338 -- Group 17\\[8pt]
Esmeralda Amado\\
Jessica Parker\\
Jonathan Su\\
Nneoma Ajawara\\
Yvan Michel Kemsseu Yobeu
}

\date{December 5, 2025}

\begin{document}
\maketitle
\onehalfspacing

\tableofcontents
\newpage

% ============================================================
\section{Version Description}

\begin{table}[h]
  \centering
  \begin{tabular}{|c|p{8cm}|c|}
    \hline
    \textbf{Version} & \textbf{Description} & \textbf{Date} \\ \hline

    1.0 & Initial SRS created for Snapshot 1. Includes purpose, scope, high-level system overview, and basic functional requirements. & Nov 28, 2025 \\ \hline

    2.0 & Updated for Snapshot 2. Added external interface requirements, expanded functional and non-functional requirements, and added legal/ethical considerations and glossary entries. & Nov 29, 2025 \\ \hline

    3.0 & Updated for Snapshot 3. Refined requirements to match early implementation, added system feature descriptions, and clarified performance/security expectations. & Dec 5, 2025 \\ \hline

  \end{tabular}
  \caption{SRS Version Description}
\end{table}

\newpage

% ============================================================
\section{Introduction}

\subsection{Purpose}
The purpose of this Software Requirements Specification (SRS) document is to
describe the functional and non-functional requirements for the AI-Powered
Triage System. This document will be updated at each snapshot checkpoint to
reflect the latest progress and system enhancements.

\subsection{Intended Audience}
This document is intended for:
\begin{itemize}
  \item CS 3338 instructor and teaching staff.
  \item Developers implementing the AI-Powered Triage System.
  \item Future maintainers extending system features or requirements.
\end{itemize}

\subsection{Scope}
The system provides healthcare workers with a secure platform that includes:
\begin{itemize}
    \item A login system for verified healthcare staff.
    \item Access to patient demographics and medical history.
    \item An AI assistant capable of summarizing patient information.
    \item The ability to update and store new visit notes.
    \item AI-assisted documentation support.
\end{itemize}

\subsection{Definitions and Acronyms}
\begin{itemize}
    \item \textbf{AI} — Artificial Intelligence.
    \item \textbf{User} — Verified healthcare staff.
    \item \textbf{Database} — Stores patient demographic and visit history.
    \item \textbf{Snapshot} — Project development milestone.
\end{itemize}

\subsection{References}
\begin{itemize}
    \item CS 3338 Final Project Instructions.
    \item HIPAA documentation (conceptual reference only).
\end{itemize}

\subsection{Document Overview}
This SRS includes:
\begin{itemize}
    \item Overall system description.
    \item Functional and non-functional requirements.
    \item External interface requirements.
    \item Legal and ethical considerations.
    \item Glossary and snapshot plan.
\end{itemize}

% ============================================================
\section{Overall Description}

\subsection{Product Perspective}
The system is a web application composed of:
\begin{itemize}
    \item A front-end interface for interacting with patients and the AI.
    \item A secure login portal.
    \item A back-end API for handling requests.
    \item A database storing patient information.
    \item An integrated AI module assisting medical staff.
\end{itemize}

\subsection{User Classes and Characteristics}
\begin{itemize}
    \item \textbf{Healthcare Staff}: Must be authenticated; understands clinical workflow.
    \item \textbf{System Administrator}: Manages permissions and configuration.
\end{itemize}

\subsection{Operating Environment}
\begin{itemize}
    \item Web browser (Chrome, Firefox, Edge, Safari)
    \item Backend server (Node.js or Python)
    \item Database (PostgreSQL or MySQL)
\end{itemize}

\subsection{Design and Implementation Constraints}
\begin{itemize}
    \item Restricted timeline due to academic semester.
    \item HIPAA compliance not fully implementable.
    \item AI features must be explainable.
\end{itemize}

\subsection{Assumptions and Dependencies}
\begin{itemize}
    \item AI API availability.
    \item Stable internet connection.
    \item Valid healthcare staff authentication.
\end{itemize}

% ============================================================
\section{External Interface Requirements}

\subsection{User Interface}
Users access the system through a secure web interface:
\begin{itemize}
  \item Login page with username/password fields.
  \item Dashboard with patient search and recent activity.
  \item Patient details page with visit history and notes.
  \item “Generate AI Summary” button.
  \item Chat-style interface for AI assistance.
\end{itemize}

\subsection{Software Interfaces}
\begin{itemize}
  \item REST API for CRUD operations.
  \item Database storing patient and visit history.
  \item AI API for summarizing patient data.
\end{itemize}

% ============================================================
\section{System Features}

\subsection{Feature F1: User Authentication}
\textbf{Description:} Ensures only authorized staff can access patient data.

\textbf{Functional Requirements:}
\begin{itemize}
    \item F1.1 System shall allow login using username and password.
    \item F1.2 Invalid login attempts must show an error.
\end{itemize}

\subsection{Feature F2: Patient Record Management}
\begin{itemize}
    \item F2.1 Search patients by name or ID.
    \item F2.2 View demographics and visit history.
    \item F2.3 Add or update visit notes.
\end{itemize}

\subsection{Feature F3: AI-Assisted Triage}
\begin{itemize}
    \item F3.1 AI shall summarize medical history.
    \item F3.2 Staff must review AI output before saving.
\end{itemize}

\subsection{Feature F4: Visit Documentation Assistance}
\begin{itemize}
    \item F4.1 AI shall help complete structured visit forms.
    \item F4.2 Final documentation must be editable before submitting.
\end{itemize}

% ============================================================
\section{Non-Functional Requirements}

\subsection{Performance}
\begin{itemize}
    \item Pages should load within 3 seconds.
    \item AI responses should return in under 5 seconds.
\end{itemize}

\subsection{Security}
\begin{itemize}
    \item Passwords must be hashed.
    \item All sensitive data must be protected.
\end{itemize}

\subsection{Usability}
\begin{itemize}
    \item Interface should require minimal training.
    \item Layout must be clear and consistent.
\end{itemize}

\subsection{Reliability}
\begin{itemize}
    \item Must support at least 20 concurrent users.
\end{itemize}

% ============================================================
\section{Legal and Ethical Considerations}

\subsection{Data Storage and Privacy}
\begin{itemize}
  \item Patient data must be stored securely.
  \item Only authenticated users may view or edit records.
\end{itemize}

\subsection{Ethical Issues}
\begin{itemize}
  \item AI suggestions are not medical advice.
  \item Staff must review AI-generated content.
\end{itemize}

% ============================================================
\section{Glossary}

\begin{table}[h]
  \centering
  \begin{tabular}{|l|p{9cm}|}
    \hline
    \textbf{Term} & \textbf{Definition} \\ \hline
    AI & Artificial Intelligence \\ \hline
    UI & User Interface \\ \hline
    API & Application Programming Interface \\ \hline
    DB & Database \\ \hline
  \end{tabular}
\end{table}

% ============================================================
\section{Snapshots}

\subsection{Snapshot 1}
\begin{itemize}
    \item Established basic project scope.
    \item Created initial SRS and SDD structure.
    \item Outlined login and patient record workflow.
\end{itemize}

\subsection{Snapshot 2}
\begin{itemize}
    \item Added database structure.
    \item Expanded external interface requirements.
    \item Clarified AI module role.
\end{itemize}

\subsection{Snapshot 3}
\begin{itemize}
    \item Updated requirements to match early implementation.
    \item Added new details for AI assistant interactions.
    \item Refined non-functional requirements.
\end{itemize}

\end{document}
