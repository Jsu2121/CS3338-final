% group17_SRS.tex  (rename the file to this when you submit)
\documentclass[12pt]{article}

\usepackage[utf8]{inputenc}
\usepackage[letterpaper,margin=1in]{geometry}
\usepackage{setspace}
\usepackage{hyperref}
\usepackage{array}

\title{Software Requirements Specification (SRS)\\
\large AI-Powered Triage System}

\author{
CS 3338 -- Group\\[8pt]
Esmeralda Amado\\
Jessica Parker\\
Jonathan Su\\
Nneoma Ajawara\\
Yvan Michel Kemsseu Yobeu
}

\date{December 5, 2025}

\begin{document}
\maketitle
\onehalfspacing

\tableofcontents
\newpage

% ============================================================
\section{Version Description}

\begin{table}[h]
  \centering
  \begin{tabular}{|c|p{8cm}|c|}
    \hline
    \textbf{Version} & \textbf{Description} & \textbf{Date} \\ \hline
    0.1 & Initial draft of SRS with core features and high-level description. & Nov 28, 2025 \\ \hline
    0.2 & Added external interface requirements, legal/ethical considerations, and glossary. & Nov 29, 2025 \\ \hline
    1.0 & Final SRS revision for CS 3338 submission. & Dec 11, 2025 \\ \hline
  \end{tabular}
  \caption{SRS Version Description}
\end{table}

\newpage

% ============================================================
\section{Introduction}

\subsection{Purpose}
The purpose of this Software Requirements Specification (SRS) document is to
describe the functional and non-functional requirements for the AI-Powered
Triage System. This document will be updated at each snapshot checkpoint to
reflect the latest progress and system enhancements.

\subsection{Intended Audience}
This document is intended for:
\begin{itemize}
  \item The CS 3338 instructor and teaching staff, for grading and review.
  \item The development team, who will implement the AI-Powered Triage System.
  \item Future maintainers, who may extend or update the system and its requirements.
\end{itemize}

\subsection{Scope}
The system provides healthcare workers with a secure platform that includes:
\begin{itemize}
    \item A user login system for verified healthcare staff.
    \item Access to patient demographic and visit history information.
    \item An AI assistant capable of reading patient records and summarizing
    relevant medical information.
    \item The ability to update and store new visit documentation.
    \item AI-assisted completion of post-visit forms using chat logs.
\end{itemize}

\subsection{Definitions and Acronyms}
\begin{itemize}
    \item \textbf{AI} — Artificial Intelligence.
    \item \textbf{User} — Verified healthcare staff.
    \item \textbf{Database} — Storage for patient demographic and visit history.
    \item \textbf{Snapshot} — Project development milestone.
\end{itemize}

\subsection{References}
\begin{itemize}
    \item CS 3338 Final Project Instructions (Canvas).
    \item HIPAA documentation (concept reference, not fully implemented).
\end{itemize}

\subsection{Document Overview}
This SRS includes:
\begin{itemize}
    \item System description.
    \item Functional and non-functional requirements.
    \item External interface requirements.
    \item Legal and ethical considerations.
    \item Glossary and snapshot planning.
\end{itemize}

% ============================================================
\section{Overall Description}

\subsection{Product Perspective}
The system is a web application composed of:
\begin{itemize}
    \item A front-end website for interacting with patients and the AI.
    \item A secure login portal.
    \item A back-end API to manage data and AI connections.
    \item A database storing patient information.
    \item An AI model integrated to assist medical staff.
\end{itemize}

\subsection{User Classes and Characteristics}
\begin{itemize}
    \item \textbf{Healthcare Staff}: Must be authenticated users; familiar with patient care workflow.
    \item \textbf{System Administrator}: Manages user permissions and system configuration.
\end{itemize}

\subsection{Operating Environment}
\begin{itemize}
    \item Web browser (Chrome, Firefox, Safari, Edge).
    \item Backend server (e.g., Node.js or Python-based).
    \item Database (PostgreSQL or MySQL).
\end{itemize}

\subsection{Design and Implementation Constraints}
\begin{itemize}
    \item Limited scope due to academic timeline.
    \item HIPAA compliance cannot be fully implemented.
    \item AI features must be explainable and safe.
\end{itemize}

\subsection{Assumptions and Dependencies}
\begin{itemize}
    \item AI API availability.
    \item Stable internet connection for users.
    \item Valid healthcare employment verification.
\end{itemize}

% ============================================================
\section{External Interface Requirements}

\subsection{User Interface}
The AI-Powered Triage System will be accessed through a secure web-based user interface:
\begin{itemize}
  \item Users log in with verified healthcare credentials.
  \item The dashboard displays a list of patients, recent visits, and triage summaries.
  \item Forms allow users to add or update patient visit documentation.
  \item A chat-style interface allows users to communicate with the AI assistant.
  \item The interface will be responsive and usable on desktop and tablet devices.
\end{itemize}

\subsection{Software Interfaces}
The system will interact with other software components and services:
\begin{itemize}
  \item A back-end REST API for creating, reading, updating, and deleting patient records.
  \item A database management system (e.g., PostgreSQL) that stores patient information and visit history.
  \item An AI model API (e.g., OpenAI or a similar provider) used to analyze patient data and generate summaries.
  \item Optional integration points for hospital systems (EHR/EMR) using secure API endpoints, if included in future work.
\end{itemize}

% ============================================================
\section{System Features}

\subsection{Feature F1: User Authentication}
\subsubsection*{Description}
Healthcare staff must log in to access patient data.

\subsubsection*{Functional Requirements}
\begin{itemize}
    \item F1.1 The system shall allow users to log in with a username and password.
    \item F1.2 The system shall validate user credentials securely.
    \item F1.3 The system shall restrict access for unauthorized users.
    \item F1.4 Login errors shall be displayed clearly.
\end{itemize}

% ------------------------------------------------------------
\subsection{Feature F2: Patient Record Management}
\subsubsection*{Description}
Users can view, search, and update patient records.

\subsubsection*{Functional Requirements}
\begin{itemize}
    \item F2.1 Users shall search for patients by name or ID.
    \item F2.2 Users shall view demographic info (name, age, gender).
    \item F2.3 Users shall view previous visit history.
    \item F2.4 Users shall add new medical visits to the patient record.
    \item F2.5 Users shall update existing visit notes.
\end{itemize}

% ------------------------------------------------------------
\subsection{Feature F3: AI-Assisted Triage}
\subsubsection*{Description}
An AI assistant helps summarize patient information and provide triage support.

\subsubsection*{Functional Requirements}
\begin{itemize}
    \item F3.1 The system shall provide an AI chat interface for staff.
    \item F3.2 The AI shall access patient history data.
    \item F3.3 The AI shall summarize previous conditions, treatments, and patterns.
    \item F3.4 The AI shall suggest severity indicators but not diagnose a condition.
\end{itemize}

% ------------------------------------------------------------
\subsection{Feature F4: Visit Documentation Assistance}
\subsubsection*{Description}
AI assists staff by generating post-visit documentation.

\subsubsection*{Functional Requirements}
\begin{itemize}
    \item F4.1 The system shall provide a structured visit form.
    \item F4.2 The AI shall extract useful information from chat logs.
    \item F4.3 Users shall edit AI-generated content before saving.
    \item F4.4 The final documentation shall be stored in the database.
\end{itemize}

% ============================================================
\section{Non-Functional Requirements}

\subsection{Performance}
\begin{itemize}
    \item NFR1: Pages shall load within 3 seconds on average.
    \item NFR2: AI responses shall generate within 5 seconds under normal conditions.
\end{itemize}

\subsection{Security}
\begin{itemize}
    \item NFR3: All sensitive data must be encrypted.
    \item NFR4: Passwords shall be stored as secure hashes.
    \item NFR5: Only authenticated users may access patient records.
\end{itemize}

\subsection{Usability}
\begin{itemize}
    \item NFR6: The user interface shall require minimal training for healthcare staff.
    \item NFR7: Information shall be presented clearly and consistently.
\end{itemize}

\subsection{Reliability}
\begin{itemize}
    \item NFR8: The system shall handle at least 20 simultaneous users.
    \item NFR9: AI responses shall be monitored to avoid harmful medical claims.
\end{itemize}

% ============================================================
\section{Legal and Ethical Considerations}

\subsection{Data Storage and Privacy}
\begin{itemize}
  \item Patient data must be stored securely with access restricted to authorized healthcare staff.
  \item All data in transit between the user interface, back end, and AI service should be encrypted (e.g., HTTPS/TLS) in a production environment.
  \item The system should follow HIPAA-like principles for handling protected health information, even if full legal compliance is out of scope for this class project.
  \item Audit logs should track who accessed or modified patient records.
\end{itemize}

\subsection{Legal and Ethical Issues}
\begin{itemize}
  \item Users must be informed that the AI assistant is a decision-support tool, not a replacement for medical judgment.
  \item Any AI-generated triage suggestion must be reviewed and confirmed by a qualified healthcare professional.
  \item Data used to train or evaluate the AI should be de-identified where possible to protect patient privacy.
  \item The team should consider bias in AI outputs and avoid using the system to make fully automated life-critical decisions.
\end{itemize}

% ============================================================
\section{Glossary}

\begin{table}[h]
  \centering
  \begin{tabular}{|l|p{9cm}|}
    \hline
    \textbf{Acronym} & \textbf{Definition} \\ \hline
    AI & Artificial Intelligence. \\ \hline
    UI & User Interface. \\ \hline
    API & Application Programming Interface. \\ \hline
    DB & Database that stores patient and visit data. \\ \hline
    SRS & Software Requirements Specification. \\ \hline
    SDD & Software Design Document. \\ \hline
  \end{tabular}
  \caption{Glossary of Acronyms}
\end{table}

\newpage

% ============================================================
\section{Snapshots (Planned Updates)}

\subsection{Snapshot 1}
\begin{itemize}
    \item Establish project scope.
    \item Build initial SRS and SDD structure.
    \item Create basic repository structure.
    \item Begin login system outline.
\end{itemize}

\subsection{Snapshot 2}
\begin{itemize}
    \item Implement database structure.
    \item Connect UI to database.
    \item Add TestRail testing document for this feature.
    \item Update SRS with new requirements.
\end{itemize}

\subsection{Snapshot 3}
\begin{itemize}
    \item Implement AI assistant feature.
    \item Connect AI to patient record database.
    \item Add TestRail document for AI testing.
    \item Update SRS requirements.
\end{itemize}

\subsection{Snapshot 4}
\begin{itemize}
    \item Finalize UI and polish system.
    \item Add future enhancement section.
    \item Add TestRail document for UI workflow.
    \item Prepare final SRS revision.
\end{itemize}

% ============================================================
\section{Appendix}
This document will be updated at each snapshot stage to reflect the current
state of the AI-Powered Triage System and its requirements.

\end{document}
